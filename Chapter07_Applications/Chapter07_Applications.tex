\section{Real-world applications {\bf{(All)}}}

RVD: My understanding of this section was rather that we collect information on all recent applications of the different methods, instead of focusing on one or two per method. Shouldn't we rather have a separate section highlighting some results for well-studied model systems (logistic map, R\"ossler, Lorenz, etc.) using all different methods, instead of having too detailed discussions of certain real-world examples?

(Update 15/02/17): I suggest splitting this section into two: (1) Numerical Examples, (2) Real-world applications. With that, we can easily proceed.

\subsection{Recurrence networks}

Although much recent work on RNs and multivariate generalizations thereof has been focused on the development of the theoretical framework and its numerical exploration using simple low-dimensional model systems, there have already been some first successful applications to characterizing system's properties from experimental or observational time series.


\subsubsection{Applications in climatology}

One important field of recent applications is paleoclimatology, which has already been taken as an illustrative example in the seminal paper by Marwan \textit{et~al.} \cite{Marwan2009}. The corresponding study was later extended to some systematic investigation of the temporal variability profile of RN-based complexity measures for three marine sediment records of terrigenous dust flux off Africa during the last 5 million years. Donges \textit{et~al.} \cite{Donges2011NPG} argued that RNs can be used for characterizing dynamics from non-uniformly sampled or age-uncertain data, since this methodological approach does not make explicit use of time information. In turn, due to the necessity of using time-delay embedding, there is implicit time information entering the analysis, which has been recognized but widely neglected in previous works. Notably, disregarding age uncertainty and sampling heterogeneity appears a reasonable approximation only in cases where the distribution of instantaneous sampling rates remains acceptably narrow. 

The results of Donges \textit{et~al.} \cite{Donges2011PNAS} pointed to the existence of spatially coherent changes in the long-term variability of environmental conditions over Africa, which have probably influenced the evolution of human ancestor species. Specifically, RN transitivity and average path length have been interpreted as indicators for ``climate regularity'' (i.e., the complexity of fluctuations as captured by the transitivity dimensions) and ``abrupt dynamical changes'', respectively. By identifying three time intervals with consistent changes of the RN properties obtained from spatially widely separated records, it has been possible to attribute the corresponding long-term changes in the dynamics to periods characterized by known or speculated mechanisms for large-scale climate shifts such as changes in the Indian ocean circulation patterns, the intensification of the atmospheric Walker circulation, or changes in the dominant periodicity of Northern hemispheric glacial cycles. Moreover, Donges \textit{et~al.} \cite{Donges2011NPG} demonstrated a good robustness of the results of RN analysis obtained in a sliding windows framework when varying the corresponding parameters (e.g., window size or embedding delay) over a reasonable range.

As another methodological step towards better understanding climatic mechanisms, we have used two speleothem records for studying interdependencies between the two main branches of the Asian summer monsoon (the Indian and East Asian summer monsoon) by means of IRNs~\cite{Feldhoff2012}. For this purpose, they selected two data sets of oxygen isotope anomalies from speleothems obtained from two caves in China and the Oman, respectively, which can be considered as proxies for the annual precipitation and, hence, the overall strength of the two monsoon branches over the last about 10,000 years. The asymmetries of the IRN cross-transitivities and global cross-clustering coefficients provided clear evidence for a marked influence of the Indian summer monsoon on the East Asian branch rather than vice versa, which is in good agreement with existing climatological theories. As a subsequent extension of this work, we emphasize the possibility of repeating the same kind of analysis in a sliding windows framework, thereby gaining information on possible temporal changes of the associated climatic patterns during certain time periods as recently revealed using correlation-based complex network analysis applied to a larger set of speleothem records from the Asian monsoon domain \cite{Rehfeld2012}.

In order to characterize dynamical complexity associated with more recent environmental variability, Lange and B\"ose \cite{Boese2012,Lange2013Book} used RQA as well as RN analysis for studying global photosynthetic activity from remote sensing data in conjunction with global precipitation patterns. Specifically, they studied 14-years long time series (1998-2011) of the fraction of absorbed photosynthetically active radiation (faPAR) with a spatial resolution of 0.5$^o$ around the Earth and a temporal sampling of about ten days, providing time series of $N=504$ data points. Their results revealed very interesting spatial complexity patterns, which have been largely, but not exclusively determined by the amplitude of the annual cycle of vegetation growth in different ecosystems.


\subsubsection{Applications in fluid dynamics}

In a series of papers, Gao \textit{et~al.} investigated the emerging complexity of dynamical patterns in two-phase gas-liquid or oil-water flows in different configurations using RN techniques. In general, multiple sensors measuring fluctuations of electrical conductance have been used for obtaining signals that are characteristic for the different flow patterns. For gas-liquid two-phase upward flows in vertical pipes, different types of complex networks generated from observational data have been proposed, among which the degree correlations (assortativity) of RNs was proven to be particularly useful for distinguishing between qualitatively different flow types \cite{Gao2009}. For oil-water two-phase upward flows in a similar configuration, the global clustering coefficient of RNs reveals a marked increase in dynamical complexity (detectable in terms of a decreasing $\hat{\mathcal{C}}$) as the flow pattern changes from slug flow over coarse to very finely dispersed bubble flow \cite{Gao2013PLA}. In case of oil-water two-phase flows in inclined pipes \cite{Gao2010PRE}, the motif distributions of RNs (specifically, the frequency distributions of small subgraphs containing exactly four vertices) revealed an increasing degree of heterogeneity, where the motif ranking was conserved in all experimental conditions, whereas the absolute motif frequency dramatically changed. The corresponding results were independently confirmed using some classical measures of complexity, which indicated increasing complexity in conjunction with increasing heterogeneity of the RN motif distributions. Finally, for characterizing horizontal oil-water flows \cite{Gao2013EPL}, RN and IRN analysis were combined for studying conductance signals from multiple sensors. Specifically, cross-transitivity was found a useful measure for tracing the transitions between stable stratified and unstable states associated with the formation of droplets.


\subsubsection{Applications in electrochemistry}

Zou \textit{et~al.} \cite{Zou2012bChaos} studied the complexity of experimental electrochemical oscillations as one control parameter of the experiments (temperature) was systematically varied. By utilizing a multitude of complementary RN characteristics, they could demonstrate a systematic rise in dynamical complexity as temperature increased, but an absence of a previously speculated phase transition \cite{wickramasinghe_chaos_2010} separating phase-coherent from noncoherent chaotic oscillations. The latter results were independently confirmed using other classical indicators for phase coherence, as well as studies of a corresponding mathematical model of the specific electrochemical processes.


\subsubsection{Applications in medicine}

Finally, there have been a couple of successful applications in a medical context. Marwan \textit{et~al.} \cite{Marwan2010Biosignal} demonstrated that the global clustering coefficients of RNs obtained from heartbeat intervals, diastolic and systolic blood pressure allowed a reliable identification of patients with pre-eclampsia, a cardiovascular disease during pregnancy with a high risk of fetal and maternal morbidity. Their results were further improved by Ram\'{i}rez \textit{et~al.} \cite{Ramirez2012,Ramirez2013} who considered combinations of various RN-based network characteristics. In a similar spirit as for cardiovascular diseases, recent results point to the capability of RN characteristics for discriminating between the EEG signals of healthy and epileptic patients \cite{Subramaniyam2013}.





%		\subsubsection{Example I: palaeoclimate data}
%		\subsubsection{Example II: soil water }

	\subsection{Visibility graphs}
		\subsubsection{Example I: sunspot numbers}
		\subsubsection{Example II: Asymmetry of sunspots}

	\subsection{Transition networks}
	\subsection{Other approaches}

	%\subsection{{\color{red} Please suggest any examples you may find appropriately here. Later, we will agree upon 1 or 2 examples which will discussed in detail for each method and mention briefly about other examples. We will include those examples that have been fully done to avoid futher serious calculations. } }
