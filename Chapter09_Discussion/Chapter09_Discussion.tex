\section{Discussion {\bf{(All)}}}

Comments by Jonathan:

- What is the added value of network methods compared to standard methods?

- Why do network methods typically seem to work better with small data sets than
standard methods? Does this have to do with $N^2$ of information units (although
not independent) being created from N data points through building an adjacency
matrix, leading to more robust statistics?

- What are common applications of network methods? Such as identifying regime
shifts / transitions / tipping points in time series, distinguishing different
dynamical regimes, analysis of dynamical system structure in phase space,
prediction of some types... What more?

We might probably start with taking these points already in the introduction (as
motivating questions beyond the methodological interest) and take them up for
more detailed discussion at the end of the review.
