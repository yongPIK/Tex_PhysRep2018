\section{Introduction {\bf{(Reik+Yong+All)}}}

With the recent increase in available computational capacities and rising data volumes in various fields of science, complex networks have become an interesting and versatile tool for describing structural interdependencies between mutually interacting units \cite{Albert2002,Boccaletti2006,Costa2007,Newman2003}. Besides ``classical'' areas of research (such as sociology, transportation systems, computer sciences, or ecology), where these units are clearly (physically) identifiable, the success story of complex network theory has recently lead to a variety of ``non-conventional'' applications. 

One important class of such non-traditional applications of complex network theory are \emph{functional networks}, where the considered connectivity does not necessarily refer to ``physical'' vertices and edges, but reflects statistical interrelationships between the dynamics exhibited by different parts of the system under study. The term ``functional'' was originally coined in neuroscientific applications, where contemporaneous neuronal activity in different brain areas is often recorded using a set of standardized EEG channels. These data can be used for studying statistical interrelationships between different brain regions when performing certain tasks, having the idea in mind that the functional connectivity reflected by the strongest statistical dependencies can be taken as a proxy for the large-scale anatomic connectivity of different brain regions \cite{Zhou2006,Zhou2007}. Similar approaches have been later utilized for identifying dominant interaction patterns in other multivariate data sets, such as climate data \cite{Donges2009b,Donges2009a,Tsonis2004}.

Besides functional networks derived from multivariate time series, there have been numerous efforts for utilizing complex network approaches for quantifying structural properties of individual time series. In the last decade, several conceptually different approaches have been developed, see \cite{Donner2011IJBC} for a recent review. One important class of approaches make use of ideas from symbolic dynamics and stochastic processes by discretizing the dynamics and then studying the transition probabilities between the obtained groups in some Markov chain-like approach \cite{Nicolis2005}. A second class are visibility graphs and related concepts, which characterize some local convexity or record-breaking property within univariate time series data \cite{Donner2012AG,Lacasa2008,Luque2009}. The latter approach has important applications, such as providing new estimates of the Hurst exponent of fractal and multi-fractal stochastic processes \cite{Lacasa2009,Ni2009} or statistical tests for time series irreversibility \cite{Donges2013EPL,Lacasa2012}. Finally, a third important class of time series networks make use of similarities or proximity relationships between different parts of a dynamical system's trajectory \cite{Donner2011IJBC}, including such diverse approaches as cycle networks \cite{Zhang2006}, correlation networks \cite{Yang2008}, and phase space networks based on a certain definition of nearest neighbors \cite{Xu2008}. One especially important example of proximity networks are recurrence networks (RNs) \cite{Donner2010PRE,Donner2010NJP,Marwan2009}, which provide a reinterpretation of recurrence plots in network-theoretic terms and are already widely applied in a variety of fields.

In this review, we summarize the current state of knowledge on the theoretical foundations and potential applications of recurrence networks. We demonstrate that this type of time series networks naturally arise as random geometric graphs in the phase space of dynamical systems, which determines their structural characteristics and gives rise to a dimensionality interpretation of clustering coefficients and related concepts. In this spirit, the rich toolbox of complex network measures \cite{Boccaletti2006,Costa2007,Newman2003} provides various quantities that can be used for characterizing the system's dynamical complexity from some exclusively geometric viewpoint and allow discriminating between different types of dynamics. Beyond the single-system case, we also provide a corresponding in-depth discussion of cross- and joint recurrence plots from the complex network viewpoint. As a new aspect not previously reported in the literature, we provide a first-time theoretical treatment of a unification of single-system and cross-recurrence plots in a complex network context. Moreover, we discuss some new ideas related to the utilization of multivariate recurrence network-based approaches for studying geometric signatures of coupling and synchronization processes.




\subsection{Nonlinear time series analysis}

A bit more on practical motivation - limitations of existing time series
analysis methods regarding certain common types of problems like detection of
subtle dynamical transitions, time-series reversibility tests, etc.? See comment
of Jonathan (added below at Discussion section)!

\subsection{Complex networks}

\subsection{Transformations from time series to network domain}
\cite{Donner2010NJP,marwan2007}

\subsection{Transformations of complex networks to time series(???)}
