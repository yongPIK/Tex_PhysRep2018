\section{Recurrence networks in phase space {\bf{(Reik)}}}

Reik to add/restructure contents from RN book chapter first
Theoretical background: Norbert to revise/add material?

\subsection{Theoretical background}

\begin{table}[t]%[H] add [H] placement to break table across pages
\caption{Summary of the definitions of vertices and the criteria for the
existence of edges in existing complex network approaches to time series analysis.}{
%\begin{ruledtabular}
\small
\begin{tabular}{llll}
\hline
Method & Vertex & Edge & Directedness \\
\hline
Proximity networks & & \\
\textit{Cycle networks} & Cycle & Correlation or phase space distance between cycles & undirected \\
\textit{Correlation networks} & State vector & Correlation coefficient between state vectors & undirected \\
\textit{Recurrence networks} & & & \\
\quad \textit{$k$-nearest neighbor networks}& State (vector) & Recurrence of states (fixed neighborhood mass) & directed \\
\quad \textit{adaptive nearest neighbor networks}& State (vector)  & Recurrence of states (fixed number of edges) & undirected \\
\quad \textit{$\varepsilon$-recurrence networks} & State (vector) & Recurrence of states (fixed neighborhood volume) & undirected \\
\hline
Visibility graphs & Scalar state & Mutual visibility of states & undirected \\
\hline
Transition networks & Discrete state & Transitions between states & directed \\
\hline
\end{tabular}
%\end{ruledtabular}
\normalsize
\label{tab:methods}}
\end{table}


		\subsubsection{Phase space and attractor reconstruction}
		\subsubsection{Recurrences and recurrence plots}
		\subsubsection{Related approaches to time series analysis}

	\subsection{Types of recurrence networks}
		\subsubsection{epsilon-recurrence networks}
		\subsubsection{k-nearest neighbor networks}
		\subsubsection{Adaptive neighbor networks}

start with contents from IJBC 2012 paper

briefly mention the results by Hui Yang on the "embeddability" of different
times using force-directed placement algorithms (Chaos paper) - discussion of
advantages and disadvantages of different neighborhood definitions  

	\subsection{Analytical theory}
		\subsubsection{Foundations: random geometric graphs}
		\subsubsection{Analytical description of epsilon-recurrence networks}

to be revised by Jonathan

	\subsection{Practical considerations}
		\subsubsection{Dependence on embedding parameters}
		\subsubsection{Choice of recurrence rate or threshold}
        
        add more recent approaches (Eroglu et al., beim Graben et al., Jacob et al)
        
		\subsubsection{Application to stochastic systems}

maybe switch 3.4 and 3.5 ?

re-organize according to common stylized network facts: scale-free degree
distributions, small-world behavior, assortativity 

	\subsection{General properties of recurrence networks}
		
        \subsubsection{Degree distributions} 
        When are RNs scale-free?
		
        \subsubsection{Transitivity and fractal dimension}
        relationship with classical R\'enyi dimensions (discussions with Grassberger)
        
		\subsubsection{Path-based characteristics}
		
        \subsubsection{Stability and robustness against noise}
        
        \subsubsection{Behavior for larger $\epsilon$}
        new papers by Jacob et al., cross-over to small-world (spatial network like) behavior

	\subsection{Multi-layer recurrence networks}

	\subsection{Inter-system recurrence networks}
		\subsubsection{Cross-recurrence plots}
		\subsubsection{Coupled networks framework}
		\subsubsection{Analytical description}
		\subsubsection{Geometric signatures of coupling}
		\subsubsection{Multi-layer multivariate recurrence networks}

	\subsection{Joint recurrence networks}
		\subsubsection{Joint recurrence plots}
		\subsubsection{Network interpretation}
		\subsubsection{Network properties and synchronization}
		\subsubsection{Algorithmic variants}

	\subsection{Other proximity-based time series networks}
		\subsubsection{Correlation networks}
		\subsubsection{Cycle networks}
